\documentclass{article}
\usepackage{mpostinl}
\begin{document}

这是一个测试文件,包含 MetaPost 代码:

\begin{mpostdef}
  input 3danim; drawing_scale:=10cm;
\end{mpostdef}

\begin{mposttex}
\usepackage[enc]{inputenc}
\end{mposttex}

\begin{mpostfig}
  numeric pi, b,c,d,e;

  pi=3.14;

  b=pi/6; c=pi/3; d=pi/5; e=pi/7;

  set_point_(1)(0,0,0); %A

  set_point_(2)(3*cosd(b),3*sind(b),0); %B

  set_point_(3)(6*cosd(c),6*sind(c),0); %C

  new_vec(v_a); new_vec(v_b); %临时向量

  vec_def_vec_(v_a,vec_I);    % v_a ← X轴单位向量 ~ı

  vec_rotate_(v_a,vec_K,d);   % 将 v_a 绕 ~k 旋转角度 d

  vec_prod_(v_b,v_a,vec_K);   % v_b ← v_a ∧ ~k

  vec_rotate_(v_a,v_b,e);     % 将 v_a 绕 v_b 旋转角度 e

  vec_mult_(v_a,v_a,4.5);     % v_a *= 4.5 (长度AD=4.5)

  vec_sum_(pnt(4),pnt(1),v_a); %D: 点1(A) + 向量v_a

  free_vec(v_b); free_vec(v_a); %释放临时向量

  % 确定I和J:

  % I = A + (AB / |AB|) * 1

  vec_diff(5,2,1);       % v5 ← AB (局部点:目标点5=点2-点1)

  vec_unit(5,5);         % v5 单位化

  vec_sum(5,5,1);        % I: 点1 + v5 (局部点5存I)

  % J = A + 4 * (AC / |AC|)

  vec_diff(6,3,1);       % v6 ← AC (局部点6=点3-点1)

  vec_unit(6,6);         % v6 单位化

  vec_mult(6,6,4);       % v6 *= 4

  vec_sum(6,6,1);        % J: 点1 + v6 (局部点6存J)
\end{mpostfig}

\end{document}

